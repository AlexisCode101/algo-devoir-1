\documentclass[12pt]{article}

\usepackage{vmargin}
\usepackage{setspace}
\usepackage[ruled, linesnumbered, french, onelanguage]{algorithm2e}
\usepackage{amsmath, amsthm, amssymb}

\title{Devoir 1}
\author{Jérémy Bouchard}
\date{\today}

\begin{document}

  \begin{titlepage}
    \doublespacing
    \centering

    UNIVERSITÉ DU QUÉBEC À CHICOUTIMI \\

    \vspace{4.7cm}

    DEVOIR 1 \\

    \vspace{4.7cm}

    PAR \\
    JÉRÉMY BOUCHARD (BOUJ08019605) \\
    ALEXANDRE LAROUCHE (LARA04119705) \\
    JEAN-PHILIPPE SAVARD (SAVJ04079609) \\
    ALEXIS VALOTAIRE (VALA) \\

    \vspace{4.7cm}

    DEVOIR PRÉSENTÉ À \\
    M. FRANÇOIS LEMIEUX \\
    DANS LE CADRE DU COURS D'ALGORITHMIQUE (8INF433)

  \end{titlepage}

  \newpage

  \newpage

  \pagenumbering{arabic}
  \onehalfspacing

  \section*{Question 1}

  \newpage

  \section*{Question 2}
  \subsection*{a)}

    Il est possible d'exprimer le pseudo-code de l'algorithme décrit dans
    l'énoncé de la question 2. Il est également possible de constater qu'il
    s'agit de l'algorithme du tri par selection. Celui-ci ce décrit de cette
    façon : \newline

    \begin{algorithm}[H]
      \KwData{\( T[1 \cdot\cdot\cdot  n] \)}
      \KwResult{\( T[1 \cdot\cdot\cdot  n] \) (Trié)}
        \For{\( j \gets 1 $ \KwTo $ n - 1 \)}{
          \( petit \gets j \)\\
          \For{\( i \gets j + 1 \) \KwTo \( n \)}{
            \If{\( T[i] < T[petit]\ \)}{
              \( petit \gets i \)\
            }
          }
          \( T[j] \leftrightarrow T[petit] \)
        }
      \caption{Pseudo-code du numéro 2}
    \end{algorithm}

  \subsection*{b)}

    Il est possible de calculer le temps d'exécution de cette algorithme
    facilement. En effet, pour chaque élément du tableau il est necessaire de
    parcourir le reste du tableau moins les éléments déjà triés. Donc, pour la
    première itération, il est nécessaire de réaliser \( (n - 1) \) comparaisons.
    Pour la deuxième itération, cela sera \( (n - 2) \) itérations et ainsi de
    suite jusqu'à ce qu'il ne reste que une seule valeur à trier. Il est donc
    possible de formaliser le temps d'exécution de cette manière :

    \[ (n - 1) + (n - 2) + (n - 3) + ... + 1 = \sum _ {i=1} ^ {n - 1} i \]

    Il ne reste qu'à trouver l'ordre de \( \sum _ {i=1} ^ {n - 1} i \). \newpage

    \begin{proof}[Temps d'exécution de l'algorithme de tri]
      \begin{align*}
        S   &= \sum _ {i=1} ^ {n - 1} i \\
            &= (n - 1) + (n - 2) + (n - 3) + ... + 1 \\
        2S  &= (n - 1) + (n - 2) + (n - 3) + ... + 1 \\
            &\qquad\,\, + 1 \qquad\;\:\, + 2 \qquad\;\:\, + 3 \;+ ... + (n - 1) \\
            &= \underbrace{(n + n + n + ... + n)}_\textrm{\( (n - 1) \) fois} \\
            &= (n - 1)(n) \\
        S   &= \frac{n(n - 1)}{2} \\
            &= \frac{1}{2}n^2 - \frac{1}{2}n
      \end{align*}
      \( D'o\grave{u} \ \mathcal{O}(n^2) \)
    \end{proof}

  \newpage

  \section*{Question 3}

  \newpage

  \section*{Question 4}

  \newpage

  \section*{Question 5}

  \newpage

\end{document}
